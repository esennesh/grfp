\section*{Relevant Background}

\begin{singlespace}
\im
I dove into my computer-science education with enthusiasm for operating
systems, programming languages, and compilers. However, I focused
too narrowly and did not perform well in general-education coursework
early in undergrad. Increased in-major coursework suited my learning
style and proved more rewarding. My time at Technion presented a new
set of challenges: an engineering-school program with high-stakes
exams and projects, and reduced classroom time. I had to re-learn
from scratch how to study and focus, especially when needing to supply
my own ``supervision signal'' besides course exercises. At Technion,
I also realized that I needed a scientific unknown towards which to
drive.
\end{singlespace}

\begin{singlespace}
\im
I made a habit of reading any programming languages papers I could,
and encountered probabilistic programming\cite{dippl}. From there,
I found the literature on computational cognitive science, and eventually
theoretical neuroscience. In that literature I saw a vivid picture:
a hard science to weave together statistics, neurobiology, and human
experience. In this field I could not only work to ship an engineered
artifact over years, but investigate exciting unknowns over decades.
\end{singlespace}

\begin{singlespace}
\im
With that goal in mind, I finished at the Technion with better overall
performance than at UMass, and followed my fiancée to Boston. I wanted
to read further into cognitive science literature, and see if I could
prepare myself for this change of field. I found work at Leaflabs,
and there was proud to be part of Google\textquoteright s Project
Ara. Its strict engineering practices helped to cement my new habits
of work and study, and built up a sense of professionalism. I had
also been fortunate enough to conduct and publish empirical research
with my adviser Yossi Gil at Technion, as well as Nicolas Koliaza
Blanchard who I met at Technion.
\end{singlespace}

\begin{singlespace}
Funding for Ara ended in September 2016, just when I had finished
reading Andy Clark\textquoteright s \emph{Surfing Uncertaint}y\cite{Clark2016-CLASUP}.
The book painted a bold portrait of the mind in which, ``{[}W{]}hat
we encounter is first and foremost a world worth acting in: a world
of objects, events, and persons, presented as apt for engagement,
permeated by affect, desire, and the rich web of conscious and non-conscious
expectations.'' I knew that I had been right all along to want a
PhD and a research career, and that I need to do it in the cognitive
sciences.
\end{singlespace}
