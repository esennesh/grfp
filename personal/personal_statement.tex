\section*{Personal Background}

\begin{singlespace}
It is well known that human beings do not receive concrete scores
or reward signals for complex behaviors such as votes in elections.
Current machine learning models remain far below human-level in many
action-oriented tasks, exactly because they require constant, immediate
reward signals\cite{Lake2016}. How can we build machines which learn
to act as fluently as people, according to human-like goals and reasons\cite{Lake2016,Railton2017}?
Recent research among the cognitive sciences has put forth a paradigm
that grips me: weaving together sensorimotor signals into perception,
knowledge, affect, and action\cite{probmods,Clark2013}. I believe
that affect should be taken seriously in artificial intelligence as
well as neuroscience. By pursuing a PhD and subsequent research career,
I am eager to blend the cognitive sciences with computation in investigating
how machines, including ``meat machines'' like us, can evaluate,
reason socially, and pursue goals.
\end{singlespace}

\begin{singlespace}
\im
Neural networks dominate machine learning today, though they have
come far from their roots in neuroscience\cite{hassabis2017neuroscience}.
I recognize the state-of-the-art task performance and industrial applications
that they bring to the field. However, I believe the prospect of human-like
machine cognition hinges on tying our computational models closely
to the cognitive sciences\cite{Lake2016}. For example, \textquotedblleft embodied
cognition\textquotedblright{} theorists argue that thought cannot
be understood without addressing its roots where the nervous system
meets the body\cite{Clark2017}. This may sound like a difficult paradigm
to implement in robotics or AI, but it has been studied\cite{pio2016active}.
In my PhD, I hope to not only gain theoretical knowledge but also
to learn about how complex human-like AI models can be made more tractable
and performant\cite{kwisthout2013predictive,Jonas2014}.
\end{singlespace}

\begin{singlespace}
\im
Northeastern's Lisa Feldman Barrett has convened a Psychologists,
Engineers, and Neuroscientists group to apply quantitative methods
to studying the predictive processing paradigm in neuroscience and
its application to the theory of constructed emotion\cite{Barrett2015,BarrettTheoryOfConstructed2017}.
Prior to officially beginning my PhD with Jan-Willem van de Meent
and Lisa Feldman Barrett as my co-advisers, I have worked with the
group on applying deep generative models to neuroimaging data. I believe,
however, that the best way to understand how a predictive brain works
is to build one, or at least build the subsystems we need to study
in detail. In my PhD, I will scale up our computational models to
the level of complexity needed to highlight subtle disagreements between
neuroscientific theories and test them rigorously.
\end{singlespace}
