\section*{Intellectual Merit}

\subsection*{Research Goal}
\begin{singlespace}
My research goal for my PhD is to model the opponent control of homeostatic
set-points in the hypothalamus as interoceptive Bayesian inference. In opponent
control, the body sends endocrine signals through the hypothalamus to the rest of
the brain, indicating when a bodily parameter has dropped too low (ie: you are
hungry), or has risen too high (ie: you feel hot and are sweating).  Either or
both of these signals can be sent at the same time in most cases, and the job
falls to the central nervous system to infer what is happening, where the
bodily parameter needs to be, to enact behaviors seeking out the appropriate
bodily state.

Existing models of reinforcement learning in AI and predictive coding in
neuroscience do not account for the brain's need to keep the body alive.
This is despite the emerging consensus around
\emph{predictive processing} as the unsupervised learning task underlying
human intelligence.  Even if we could collapse those homeostatic sensory
signals into a single number, termed ``reward'', we would have the problem
that ``deep reinforcement learning does not work yet''\cite{rlblogpost}.  Even
the more computationally efficient methods for optimizing behavior, such as model
predictive control, still require the task to be specified as a scalar
cost\cite{6386025}.  I believe that formalizing the signals from which the brain infers
and predicts the body's needs can lead to much more efficient and tractable machine
learning models for active tasks.
\end{singlespace}

\begin{singlespace}
The human brain is tasked with monitoring a variety of feedback signals, from
blood-sugar concentration to core body temperature.  The state of these
variables in the moment can be called a \emph{homeostatic} state, and
keeping the homeostatic state constant is called homeostasis.  We now know,
however, that the brain accomplishes a harder task: \emph{allostasis}, steering
internal variables and modulating rewards over time, according to the physiological
needs of the moment\cite{Sterling2012,Kleckner2017}.  This may be accomplished by the cooperation
of the predictive brain with opponent control signals from the body whose
intensity and thresholds in the parameter space vary as bodily needs do\cite{Morville2018a}.
\end{singlespace}