\section*{Intellectual Merit}

\subsection*{Research Goal}
\begin{singlespace}
One important role of the brain is to interpret and integrate signals from the outside world (exteroceptive
information) and from inside the body (interoceptive information) to guide action.  There is
a growing consensus in neuroscience that instead of asking, ``what is this sensory input?'',
the brain infers, ``what is this most like?'', and uses similar past experience to dynamically make
present inputs meaningful \cite{Bar2009,Friston2010,BarrettTheoryOfConstructed2017}.
In psychology, a \emph{category} is a group of instances sharing a functional similarity
within a context (e.g., \cite{Grill-Spector2014,Greene2016}).
Therefore, an understanding of how the brain works requires solving the puzzle of how it categorizes
present inputs by constructing categories from past experiences, in a specific situation or context.
What functional purpose or criterion is the brain using to spontaneously construct categories?  An emerging perspective
in neuroscience suggests the brain primarily anticipates the energy needs of the body and attempts to efficiently
meet those needs before they arise, a process called \emph{allostasis} \cite{Sterling2012}.
For additional references, see \cite{sterling2015principles,Kleckner2017}.  The functional similarity that supports dynamic category construction, then, may be grounded in efficient energy regulation of the body \cite{Barrett2015,BarrettTheoryOfConstructed2017}.
\end{singlespace}

\begin{singlespace}
To test this hypothesis, I will use hierarchical Bayesian topic modeling to study an established phenomenon in psychology called \emph{emotional granularity}, which
\cite{Kashdan2015} has been described as a brain-based category construction problem \cite{BarrettTheoryOfConstructed2017}.  The brain categorizes interoceptive and exteroceptive
signals into bread categories, such as ``pleasantness'' and ``unpleasantness'', which do not
guide the physical systems of the body in a precise way, or into narrower ones, such as
``irritation'' or ``despondence'' which allow for more precise action planning.  Individuals
with greater emotional granularity tend to achieve greater educational and professional
success, enjoy better physical and mental health, and recover from illness more quickly (e.g., \cite{hoyt2013cancer,Kashdan2015}; for an extended discussion,
see \cite{barrett2017emotions}).  In my PhD, I will study how to improve the measurement
of emotional granularity using hierarchical topic models, applied to existing and forthcoming data sets,
and by integrating autonomic nervous system endpoint measurements into those models.
\end{singlespace}
