% I know the \begin{singlespace} things are annoying... But I couldn't get
% LaTeX to actually obey this specification when used across multiple
% paragraphs. There may be an easy way to fix this.

\subsection*{Research Plan}
\begin{singlespace}
In my research, I can first extend recent models of \emph{active inference} (from
neuroscience) to handle opponent control signals, as well as environments such as
the game Frostbite in which deep reinforcement learning continues to underperform
human learning speeds\cite{Cullen2018,Lake2016}.  Design constraints and agent goals
can already be specified via probabilistic programs when given a symbolic
representation of the underlying design
space\cite{botvinick2012planning,Yeh:2012:SOW:2185520.2185552,Ritchie:2015:CPM:2809654.2766895}.
A neuroscientifically-grounded model of motivational control will endow agents with
a human-like \emph{inductive bias}: learn to survive, and explore what you can
affect right now.  Opponent control could also potentially be used to encode positive
feedback loops and pure rewards.
\end{singlespace}

\begin{singlespace}
The resulting models of time-varying allostatic set-points can also be checked in
physiological and neuroimaging studies.  Analyzing simultaneous responses in the
central and autonomic nervous systems while subjects are challenged can allow us
to measure the flexibility of mental activity, and we hypothesize this can act as
a measure of successful opponent control of autonomic systems under time-varying loads.
These studies will not only provide an opportunity to falsify our current generative
model of opponent control, but to notice patterns in data which could lead to an
improved replacement model.
\end{singlespace}

\section*{Broader Impacts}
\begin{singlespace}
My PhD appointment at Northeastern University is unique in providing me the resources
to investigate how the brain maintains the body from a computational perspective. My
primary adviser Jan-Willem van de Meent and co-adviser Lisa Feldman Barrett are founding
principal investigators of the Psychologists, Engineers, and Neuroscientists (PEN) group, and
Professor Barrett is the President of the Association for Psychological Science.
The PEN group aims precisely to bring engineering and computer-science expertise to bear on
the study of the brain as an organ of
allostasis\cite{Kleckner2017}.
Beyond just implementing a neuroscience-inspired AI idea and testing it in an
environment such as the OpenAI Gym, I have the opportunity to extrapolate from a
hypothesis about allostasis to predictions about human and animal subjects, and then
test those predictions against real studies.  That initial implementation of brain-inspired
AI has been done and tested\cite{Cullen2018,Penny}.  Now, in the PEN group, I believe we
can advance most efficiently by subjecting computational ideas to rigorous neuroimaging,
behavioral, and physiological studies.  A clear idea of how the brain coordinates
allostasis will not only provide fertile soil for AI investigations: dysfunctional
allostatic processing is implicated in many of today's most pressing health issues.
Depression, anxiety, addiction, and obesity can all arise from malfunctions in
the brain's core allostatic mechanisms\cite{Rosen2004,Stephan2016}.
\end{singlespace}
