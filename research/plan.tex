% I know the \begin{singlespace} things are annoying... But I couldn't get
% LaTeX to actually obey this specification when used across multiple
% paragraphs. There may be an easy way to fix this.

\subsection*{Research Plan}
\begin{singlespace}
In published studies, emotional granularity is measured by applying simple covariance metrics to self-reports
of emotional experience in which study participants rate the intensity of their experiences
according to a set of emotion adjectives at a given time
point \cite{10.3389/fnhum.2017.00133}. In contrast, a topic
model can measure the granularity with which an ideal observer would report emotion experiences and examine the extent to which these categories have some physical basis in autonomic nervous system measurements or even brain activity.
\end{singlespace}

\begin{singlespace}
I chose to use hierarchical Bayesian topic models to study dynamic category construction in
the predictive mind due to the functional correspondence between the two.  Hierarchical topic
models follow a history of successful applications to other problems \cite{7783636}.  They
provide the best fit to studying emotional granularity, out of all machine learning models,
because they can \emph{discover} nested categories on a person-by-person basis, rather than
imposing folk-psychological categories not respected by biological and behavioral data \cite{clark2017multivoxel,Gendron2018,Siegel2018}.
They can also capture how different features lead to different categories
for each participant, and can describe the \emph{concentration} of data (or nested categories) at every level
into more or fewer categories.  This will allow us to investigate granularity at multiple levels, within and
across participants.
\end{singlespace}

\begin{singlespace}
I will primarily be guided by Dr. Jan-Willem van de Meent, an expert in probabilistic
programming \cite{2018arXiv180910756V}.  I am also co-advised by Dr. Lisa Feldman Barrett,
an internationally recognized psychologist and neuroscientist who discovered emotional
granularity and who has pioneered the emerging constructionist paradigm for discovering
the nature of emotion \cite{BarrettTheoryOfConstructed2017}.  I will begin my investigation
using an existing dataset consisting of subjects' reports of their momentary emotional
states, and measured physiological endpoint states across several weeks.
\end{singlespace}

\begin{singlespace}
My longer-range plan is to build a laboratory studying how the body corrects the brain's
category construction dynamics.  I hypothesize that the brain, in maintaining allostasis,
\emph{defends} certain physiological states within narrow ranges
conducive to survival (and reproduction), via ongoing behavior\cite{Sterling2012}.  Interoceptive
signals related to defended parameters should show differences from exteroceptive sensory signals, and I will continue the work I begin in my PhD by characterizing these.
\end{singlespace}

\section*{Broader Impacts}
\begin{singlespace}
Drs. van de Meent and Barrett are founding principal investigators of the Psychologists, Engineers,
and Neuroscientists (PEN) group, dedicated to building a multidisciplinary understanding of brain
function as predictive regulation \cite{Sterling2012,Kleckner2017}.  I am in the first cohort of
PEN's graduate trainees to receive fully multidiscplinary training in applying methods from 
formal sciences such as computing and Bayesian statistics to address the problems of
psychological science and neuroscience.  In summer 2018 I assisted psychologist
colleagues in learning Python and the PsyNeuLink block-modeling framework, and I will also help
train interested psychologists and neuroscientists at PhD and post-doc levels computational modeling
techniques.
\end{singlespace}

\begin{singlespace}
Since the models I build require reusable software, I am also making a technological
contribution.  I have presented a library of modeling and inference
combinators for Probabilistic Torch at the International Conference on Probabilistic Programming
\cite{Sennesh2018}.
\end{singlespace}
