% I know the \begin{singlespace} things are annoying... But I couldn't get
% LaTeX to actually obey this specification when used across multiple
% paragraphs. There may be an easy way to fix this.

\section*{Research Plan}
\begin{singlespace}
\im
I will begin my work by replicating the work of Cullen and colleagues on
using active inference in the OpenAI Gym environments, such as the ``DOOM''
task\cite{Cullen2018}, re-engineering active inference in a probabilistic
programming framework.  From there, my main work will consist of using that
framework to computationally formulate and test hypotheses of how the
interoceptive brain performs the ``body budgeting'' required for allostasis.
Hypotheses at hand include ``mere'' homeostasis, defined as permanent
interoceptive set-points encoded as ``hyperpriors'' in the brain's model of
the body\cite{Morville2018a}; dynamic flow networks for physiological
resources; and brain contributions to integral rein control via hormones\cite{SAUNDERS1998}.  This would be one
of the first initiatives to model more than just the brain's own energy
budgeting, but how the brain uses interoception to perform allostatic
control\cite{Sterling2012,Sterling2015,Christie2015}.  Our models will
also give us the opportunity to test diachronic, behavioral, and teleological
concepts of allostasis\cite{Corcoran2017}.
\end{singlespace}

\begin{singlespace}
\im
Beyond just modeling allostasis in the central nervous system, scaling
up active inference will allow us to specify the goals for robotics and AI
control tasks as composeable probabilistic programs.  Homeostatic control has
been attempted in a machine learning agent, but never as a scaled-up way
of controlling a sizeable task environment\cite{Penny}.  In environments such
as OpenAI Gym video games, we can endow agents with ``goals'' related to
``interoceptive'' in-game quantities, such as health points or usable
resources, as well as externally-directed goals such as reaching an exit
within the game level.  Encoding these goals as probabilities and optimizing
them with active inference will allow experimenters to compose goals by
forming joint distributions from conditional ones.
\end{singlespace}

\begin{singlespace}
\im\bi
I will be conducting my research as part of the Psychologists, Engineers,
and Neuroscientists Group (PEN) convened by Lisa Feldman Barrett to model
the predictive brain and the theory of constructed emotion\cite{Barrett2015,BarrettTheoryOfConstructed2017}.
As part of this colloboration, I am being co-advised by both Jan-Willem van
de Meent, an expert in probabilistic programming, and Professor Barrett as
part of her Interdisciplinary Affective Science Lab.  Since the Lab performs
both purely investigative and clinical research, I will be able to test my
models of allostasis and core affect against clinical data, and apply it
clinically.  Dysfunctional interoceptive inference and visceromotor planning
are implicated in many of today's most pressing health issues: depression,
addiction, and obesity can all arise from malfunctions in the brain's core
allostatic and interoceptive mechanisms that construct affect\cite{Stephan2016}.
\end{singlespace}
